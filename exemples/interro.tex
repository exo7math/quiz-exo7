\documentclass[a4paper,10pt]{article}
\usepackage[french]{babel} % réglage de la langue
\usepackage[utf8]{inputenc} % permet de taper des caractères accentués dans le fichier source
\usepackage[T1]{fontenc} % permet d'afficher correctement les caractères accentués, avec césures etc
\usepackage{lmodern} % rétablissement des polices vectorielles
\usepackage[margin=2cm]{geometry}
\usepackage{amssymb,amsthm,amsmath,mathrsfs,stmaryrd,multicol,comment} 
\usepackage{tikz}
%\usepackage[francais,bloc,ordre]{automultiplechoice} % option 'bloc' pour ne pas changer de page dans une question, et 'ordre' pour ne pas permuter les réponses.


\newcommand\pad[1]{\ifnum #1 < 1000 0\fi \ifnum #1 < 100 0\fi \ifnum #1 < 10 0\fi #1} % un peu moche mais bon : sert à ajouetr la bonne quantité de zéros devant un numero de question

\newenvironment{reponses}{\begin{center}}{\end{center}}
\newcommand{\mauvaise}[1]{\hspace{.5cm}#1\hspace{.5cm}}
\newcommand{\bonne}[1]{\hspace{.5cm}\fbox{#1}\hspace{.5cm}}

% si on ne veut pas afficher la correction, commenter la ligne précédente et décommenter la ligne suivante:
%\newcommand{\bonne}[1]{\hspace{.5cm}#1\hspace{.5cm}}



% packages et macros pour questions Maxime :
\usepackage{mathdots}
\newcommand{\R}{\mathbb R}
\newcommand{\C}{\mathbb C}
\newcommand{\N}{\mathbb N}
\newcommand{\Z}{\mathbb Z}
\newcommand{\eq}[1]{\underset{#1}{\sim}}
\newcommand{\cvg}[1]{\xrightarrow[#1]{}}
\DeclareMathOperator{\tr}{Tr}
\DeclareMathOperator{\id}{Id}
\DeclareMathOperator{\rg}{rg}
\DeclareMathOperator{\im}{Im}
\DeclareMathOperator{\Mat}{Mat}

\theoremstyle{definition}
\newtheorem{question}{Question}



\begin{document}

NOM, PRÉNOM : 

\begin{center}
\huge{Interro 3 : révisions}\\
\large{CONSIGNE : entourez la bonne réponse.}
\end{center}

\setlength\columnsep{1cm}
\begin{multicols}{2}

\section{Arithmétique}
\foreach \n in {92,95,107,109,97,99,100} {%
	\begin{question}%
	\input{../latex_amc/10\pad{\n}.tex}%
	\end{question}%
}%

\section{Dérivées}
\foreach \n in {239,240,244,245,255,261} {%
	\begin{question}%
	\input{../latex_amc/10\pad{\n}.tex}%
	\end{question}%
}%

\columnbreak

\section{Valeur absolue}
\foreach \n in {1, 2,3,6,12,19} {%
	\begin{question}%
	\input{../latex_amc/10\pad{\n}.tex}%
	\end{question}%
}%

\section{Racines carrées}
\foreach \n in {748,749,758,761,767,775} {%
	\begin{question}%
	\input{../latex_amc/10\pad{\n}.tex}%
	\end{question}%
}%


\end{multicols}

\end{document}