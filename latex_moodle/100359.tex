\begin{truefalse}{q-359}
<b>Énoncé</b> : déterminer le domaine de définition de $\sqrt{x^2-5x+6}$.<br> <b>Solution rédigée à évaluer :</b><br>  «Soit $x\in\mathbb{R}$.  L'expression $\sqrt{x^2-5x+6}$ est bien définie ssi $x^2-5x+6$ est positive. Le discriminant de ce trinôme vaut $\Delta = 25-24=1$, les deux racines sont $2$ et $3$ et son coefficient dominant est positif. Le domaine de définition de $\sqrt{x^2-5x+6}$ est donc $]-\infty,2]\cup[3,+\infty[$.»
\item* Vrai
\item Faux
\end{truefalse}

