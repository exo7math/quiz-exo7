\begin{truefalse}{q-55}
$f:\left\{\begin{array}{ccc}\mathbb{N} & \to & \mathbb{N} \\n & \mapsto & 2n\end{array}\right.$ est injective.
\item* Vrai
\item Faux
\end{truefalse}

